\chapter{Introduction}

\section{Research Question}

Aim of this paper is to compare existing and common LOD frameworks to give TU Wien a decision guidance for choosing one. The concrete research question is as following:

\textbf{RQ:} \textit{How do common LOD frameworks compare against each?}
\begin{enumerate}
\item \textbf{RQ1:} What are existing frameworks?
\item \textbf{RQ2:} What are criteria to compare frameworks?
\item \textbf{RQ3:} How do they compare against each other?
\item \textbf{RQ4:} What can be a solution for TU Wien?
\end{enumerate}

The conducted work is a follow-up work of a previous study~\cite{baronyai_publishing_2016} done by the author and will build up on it.

\section{Methodology}

The work was done as a literature study. First a definition of the term "framework" has to be found for this paper in order to achieve the research question. Then a range of existing L(O)D projects and application was investigated, to extract used technologies from them. From this, candidates then were retrieved and classified. After excluding and filtering some of the candidates, they were compared in four criteria groups: Criteria from the above mentioned study, usability, data formats and the Linked Data Publishing Checklist.

\section{Structure of this Paper}
This paper starts with the definition of the used methodology in chapter~\ref{chap:methodology}, where the term "framework" will be defined for this paper (section~\ref{def:framework}), the process of the literature study (section~\ref{meth_study}) and the classification (section~\ref{classification}) system will be defined.

In chapter~\ref{overview} the found solution will be reviewed and described as well as the excluded candidates (section~\ref{excluded}).

In chapter~\ref{criteria} the criteria as well as there according scala will be defined.

The final comparison will be done in chapter~\ref{comparison}, the summary of it can be found in section~\ref{comp_summary}.

The last research question, RQ4, will be answered in chapter~\ref{ch:tuwien}, investigating, which of the proposed solutions may be suitable for which situation at TU Wien.

The last chapter, \ref{ch:summary}, describes the overall summary and future work.