\chapter{Introduction}

The Semantic Web is getting more and more popular and with it the need for ways to 
publish Linked (Open) Data. In order to cover these needs, a lot of different 
tools, frameworks and solutions were developed, some covering the whole process of 
publishing LD, others covering only steps in the process of it. These led to a 
vast diversity but also a big number of options. In order to develop a LD project, 
the responsible stakeholders now have to choose among these many options, which is 
tedious and time-consuming process]. There is a clear lack for a comparison 
framework of existing LD solutions. .

\section{Research Question}

Aim of this thesis is to compare existing and common LD solutions to give 
responsible stakeholders at TU Wien a decision guidance for choosing one. The 
concrete research question is as follows:

\textbf{RQ:} \textit{What is a suitable evaluation framework for comparing LD solutions?}
\begin{enumerate}
\item \textbf{RQ1:} What are the most popular LD Solutions?
\item \textbf{RQ2:} What are criteria to compare LD solutions? How can an Evaluation Framework for comparing LD solutions look like?
\item \textbf{RQ3:} Is the evaluation framework suitable for comparing LD solutions? 
\item \textbf{RQ4:} How can the Evaluation Framework and the results of the comparison help as guideline at TU Wien?
\end{enumerate}

The conducted work is a follow-up work of a previous study (see ~
\cite{baronyai_publishing_2016}) done by the author and will build up on it.

\footnotetext{\textbf{Note}: The term \textit{solutions} is used here on purpose 
to abstract terms like \textit{framework}, \textit{tool} or \textit{all-in-one 
solution}. Using only the term \textit{framework} would lead to problems and cut 
out other options. For a more detailed discussion see section~
\ref{def:framework}}.

\section{Methodology}

The work was done as a literature study. First a discussion of the term 
"framework" was done in order to achieve the research question. Then a range of 
existing LD projects and application was investigated, to extract used 
technologies from them. From this, candidates were retrieved and classified. After 
excluding and filtering some of the candidates, they were compared in four 
criteria groups: Criteria from the above mentioned study, usability, data formats 
and the Linked Data Publishing Checklist.

\section{Structure Of This Thesis}
This thesis starts with a state of the art section in which similar comparision 
will be described. The next chapter~\ref{chap:methodology} is the definition of 
the used methodology, where a discussion of the term "framework" will be presented 
for this thesis (section~\ref{def:framework}), the process of the literature study 
(section~\ref{meth_study}) and the classification (section~\ref{classification}) 
system will be defined.

In chapter~\ref{overview} the found solution will be reviewed and described as 
well as the excluded candidates (section~\ref{excluded}). In chapter~
\ref{criteria} the criteria as well as their according scala will be defined. The 
final comparison will be done in chapter~\ref{comparison}, the summary of it can 
be found in section~\ref{comp_summary}.

The last research question, RQ4, will be answered in chapter~\ref{ch:tuwien}, 
investigating, which of the proposed solutions may be suitable for which situation 
at TU Wien.

The last chapter, \ref{ch:summary}, describes the overall summary and future work.