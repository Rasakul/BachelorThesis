\chapter{Methodology (RQ2 \& RQ3)}
\todo{Enter your text here.}
\section{Definitions for this paper}
\subsection{Framework}~\label{def:framework}
\section{About the difficulty of comparing frameworks}
\section{Criteria}

\begin{itemize}
\item Maintainability
How much effort needs the maintenance? 

\item Data quality 
\begin{itemize}
\item Data freshness (ability to handle new data)
\item Flexibility (of ontology) (deal with heterogenous and/or legacy data)
\end{itemize}

\item Usability: (adopted from Abran et.al.~\cite{abran2003usability} to fit)
\begin{itemize}
\item Effectiveness (How well do the users achieve their goals using the system?)
\item Efficiency (Time to achieve one task, complexity to handle)
\item Satisfaction
\item Security
\item Learnability (Documentation)
\item Performance
\end{itemize}

\item Available data formats (HTML, Relational Databases, Wrapping Existing Application or Web APIs, XML, Tables/Spreadsheets)

%LD book
\item Linked Data Publishing Checklist (from Heath et.al.~\cite{heath2011linked})
\begin{itemize}
\item Does your data set links to other data sets?
\item Do you provide provenance metadata?
\item Do you provide licensing metadata?
\item Do you use terms from widely deployed vocabularies?
\item Are the URIs of proprietary vocabulary terms dereferenceable?
\item Do you map proprietary vocabulary terms to other vocabularies?
\item Do you provide data set-level metadata?
\item Do you refer to additional access methods?
\end{itemize}
\end{itemize}