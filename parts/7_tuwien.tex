\chapter{A Case Study Of Applying The Evaluation Framework At TU Wien (RQ4)}\label{ch:tuwien}

In this section, the thesis tries to answer research question 4: What a LD 
solution for TU Wien might look like.

\section{General Considerations}
In order to answer the question, it is important to give a context, since it is 
difficult respectively impossible to give a general, generic solution. In the 
following sections, three possible scenarios will be proposed, which can be a 
possible use case at TU Wien. For each scenario the requirement will be defined 
and a solution assigned.

From the previous study~\cite{baronyai_publishing_2016} conducted by the author 
the following requirements are important for the TU stakeholders:

\begin{itemize}

\item \textbf{Maintainable}: \\
A LD application at TU Wien has to be easily maintainable. Citing stakeholders 
from the previous study, it "simply has to work without having to care about 
continuously".

\item \textbf{Fresh Data}: \\
It was important for the stakeholders, that the provided data are up-to-date.

\item \textbf{Legacy Data}: \\
A possible solutions for TU Wien has necessarily handle legacy data since this are 
the kind of data, that are candidates for publishing.

\end{itemize}

These requirements are therefore valid for each following scenario, meaning that 
criteria group 1 is higher weighted than the other groups. A solution, that does 
not full-fill one of this criteria, will not be take account of at all.

\section{Situations \& Requirements}

As mentioned, to answer the research question, the context needs to be specified. 
This will be done by analyzing the stakeholders (already done by the previous 
study). These findings will then be combined to scenarios.

\subsection{Stakeholders}

\textbf{Researcher} are interested in an easy access to the publications of the 
university and publications in general as well as an easy access to resources for 
their research subject. They need to \emph{use} the data without further work in 
order to focus on their actual daily work. That also applies if the research is 
teaching at the university: in order to use an application, either by accessing it 
or providing data to it, it would be rejected by the stakeholders, if it means 
significant more work. But since they are used to technical work, the inhibition 
threshold to use a LD application is lower than for the administration staff.

\textbf{Administration staff} is more a subject of data source than of data user 
since they are holding a lot of interesting data. But on the other hand, the 
inhibition threshold is higher than for the researcher since the technical 
experience is at average lower. Therefore the argument of an "easy-to-use" of the 
previous stakeholder group is even more valid here.

\textbf{Students} are mainly data consumer and therefore more concerned with the 
frontend of a possible application. This group can build application that are 
consuming the published data.

\subsection{Situations}

In the following sections will describe possible scenarios at TU Wien, how and 
which data are going to be published. There are two base scenarios: publishing a 
single data source and building a platform with various data sources. The second 
one will be additional split.

\subsubsection{Situation 1: Specialized Single Solution}

Only one or a small number of data source(s) needs to be published. The 
application has to be specific done for this data set. The project is small scaled 
and can be done quickly. A possible data source could be the publication database. 
This scenario can be used as a demonstration, how to \emph{consume} Linked Open 
Data, but it is not advisable to use this approach if any other data source might 
be published in the future, since this scenario is \emph{not} scalable

\textbf{Requirement}
\begin{itemize}
\itemsep0pt
\item small scale
\item small number of data sets
\item specialized solution
\item simple
\end{itemize}

\subsubsection{Scenario 2: Function-rich Platform}

A comprehensive platform is needed, which covers a variety of data of different 
formats, including relational data as well as semi-structured data like XML. The 
project is medium scaled, the platform shall be able to handle data TU wide and is 
not only for internal purpose but also for external usage. Additional, the 
platform has to offer meta data, documentation and performance data. The solution 
has to be easy to use and implement. Licensing or Open Source does not play a 
role. 

\textbf{Requirement}
\begin{itemize}
\itemsep0pt
\item various kinds of data sets (in number and formats)
\item medium scale
\item platform with additional features
\item easy to use/implement
\end{itemize}

\subsubsection{Scenario 3: Complete Controlled Platform}

Similar to scenario 2, it is necessary to build platform combining different kinds 
of data sources. But in contrast, full control over the platform and the 
technology stack is necessary and no licensing is wanted, an Open Source Solution 
is necessary. In exchange, the requirement "easy to use/implement" is relaxed, but 
still relevant.

\textbf{Requirement}
\begin{itemize}
\itemsep0pt
\item various kinds of data sets (in number and formats)
\item medium scale
\item platform with additional features
\item no licensing
\item Open Source
\end{itemize}

\section{Proposed Solutions}

\subsection{Scenario 1: Specialized Single Solution}

There are two solution thinkable:

\paragraph{Option 1: D2RQ} can be a solution for this situation. It provides a 
simple mapping of a given data set and a method of publishing with the D2R Server. 
The project can be fast implemented and be ready. Other data sets may be added. 
Since D2RQ only handles relational data, this approach is not suitable for semi-
structured data.

\paragraph{Option 2: A combination of Jena/RDF4J with Euclid and LDIF} can be used 
to publish he given data. The Euclid architecture (assuring maintainability) can 
be used as a blueprint to implement an application using either the Apache Jena or 
Eclipse RDF4J framework. If a mapping is needed, LDIF can be used additionally. 
This solution is far more extensive than using D2RQ, but on the other hand more 
flexible and customizable. The scope can be controlled by defining the limits, 
specially by economizing the UI, but be aware that this approach can result in an 
lesser usability and the scope might be too big for this situation.

\subsection{Scenario 2: Function-rich Platform}

For this scenario is the \textbf{Information Workbench} suitable. The tool can 
integrate different kinds of data sources, combine the access points to a platform 
and providing additional features over modules. The implementation can be done 
relatively simple and fast. However, it needs a licensing if used outside an 
educational scope.

\subsection{Scenario 3: Complete Controlled Platform}

For this scenario no all-in-one solution of the proposed tools is suitable, 
instead the platform has to be developed by using a framework like \textbf{Jena or 
RDF4J} in combination with the \textbf{Euclid Project architecture}, similar to 
the solution proposed situation 1. If a mapping to an unified vocabulary is 
needed, \textbf{LDIF} can also be used. This project will probably consume more 
time and will have a bigger scope than the solution of situation 2. But on the 
other hand, this approach offers more customization options.

\section{Summary}

\begin{table}[hbtp]
\centering
\resizebox{\textwidth}{!}{%
	\begin{tabular}{|l|l|l|}
		\hline
		\rowcolor[HTML]{EFEFEF} 
		\textbf{Scenario} & \textbf{Requirement} & \textbf{Solution} \\ \hline
		%%
		\textbf{\begin{tabular}[c]{@{}l@{}}Specialised \\ Single Solution\end{tabular}} & 
		\begin{tabular}[c]{@{}l@{}}
			-) small scale \\ 
			-) small number of data sets \\ 
			-) specialized solution \\ 
			-) simple
		\end{tabular} & 
		\begin{tabular}[c]{@{}l@{}}
			-) Option 1: D2RQ \\ 
			-) Option 2: Jena + Euclid \\ (+ LDIF)
		\end{tabular} \\ \hline
		%%
		\textbf{\begin{tabular}[c]{@{}l@{}}Function-rich \\ Platform\end{tabular}} & 
		\begin{tabular}[c]{@{}l@{}}
			-) various kinds of data sets \\ (in number and formats) \\ 
			-) medium scale \\ 
			-) platform with additional features \\ 
			-) easy to use/implement
		\end{tabular} & 
		Information Workbench \\ \hline
		%%
		\textbf{\begin{tabular}[c]{@{}l@{}}Complete \\ Controlled \\ Platform\end{tabular}} & 
		\begin{tabular}[c]{@{}l@{}}
			-) various kinds of data sets \\ (in number and formats) \\ 
			-) medium scale \\ 
			-) platform with additional features \\ 
			-) no licensing \\
			-) Open Source
		\end{tabular} & 
		\begin{tabular}[c]{@{}l@{}}
			Jena/RDF4J + Euclid \\ (+ LDIF)
		\end{tabular} \\ \hline
	\end{tabular}%
}
\caption{Summary of Scenarios \& proposed solutions for TU Wien}
\label{tu_scenarios_overview}
\end{table}

Taking a deeper look at the proposed situations and solutions, it can be seen, 
that it is most likely to use solutions like Jena or RDF4J in combination with a 
meta architecture like Euclid and utilization tools like LDIF. Tools like D2RQ are 
limiting and the Information Workbench needs a licensing. Looking at the 
popularity charts of sites like DB-Engines~\footnote{\url{https://db-engines.com/
en/ranking_trend/system/Jena\;RDF4J}}, it currently seems like Jena is quite more 
popular.

This proposed solution is of course not complete, there are similar tools to Jena 
and RDF4J, like MarkLogic~\footnote{\url{http://www.marklogic.com/}} or Virtuoso~
\footnote{\url{https://virtuoso.openlinksw.com/}}. But it can be stated, that this 
\textit{kind} of solution is the most suitable of the proposed \textit{kinds} of 
solutions.