\begin{abstract}

Along with the rising popularity of the Semantic Web and Linked Open Data, there are many tools and frameworks present which support building LD applications. It is hard to choose an approach among the many available options without guide of comparison. This is especially true for TU Wien, since the university has to fulfil the needs of very different stakeholders at the same time while trying to integrate a LD solution into existing structures and workflows. This paper aims to give an overview of different kind of LD tools and frameworks and compare them among each other to give stakeholder at TU Wien a guideline for a future LD project at the university. In order to do that, a literature study was conducted to choose candidates for the comparison. These 8 candidates (Euclid project, LUCERO, Linked Data book, D2RQ Platform, Information Workbench, LDIF, Eclipse RDF4J and Apache Jena) were then compared with a new comparison framework, developed for this paper. Problems raised due the different nature of the investigated candidates (frameworks VS tools). At the end this thesis, the results are analysed for a usage at TU Wien.

\end{abstract}