\begin{abstract}

Along with the rising popularity of the Semantic Web and Linked Open Data, there are many tools and frameworks present which support building LD applications. It is challenging to choose an approach among the many available options without a guide of comparison. This is especially true for TU Wien, since the university has to fulfill the needs of very different stakeholders at the same time while trying to integrate a LD solution into existing structures and workflow. This paper aims to give an overview of various LD tools and frameworks and compare them among each other to give stakeholders at TU Wien a guideline for a future LD project at the university. 

In order to do that, an evaluation framework based on four families of comparison criteria was designed. It was then validated in two ways: First, it was applied to compare 8 LD solutions (Euclid project, LUCERO, Linked Data book, D2RQ Platform, Information Workbench, LDIF, Eclipse RDF4J and Apache Jena), which were found by conducting a literature study. Second, based on this comparison, a suitable solution was recommended to the use case of TU Wien. Our use of the framework in the TU Wien context, indicated that it makes the selection of an LD candidate well-informed and much faster.

\end{abstract}