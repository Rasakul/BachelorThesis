% Copyright (C) 2014-2016 by Thomas Auzinger <thomas@auzinger.name>

\documentclass[draft,final]{vutinfth} % Remove option 'final' to obtain debug information.
%\documentclass[draft,final]{vutinfth} % Remove option 'final' to obtain debug information.

% Load packages to allow in- and output of non-ASCII characters.
\usepackage{lmodern}        % Use an extension of the original Computer Modern font to minimize the use of bitmapped letters.
\usepackage[T1]{fontenc}    % Determines font encoding of the output. Font packages have to be included before this line.
\usepackage[utf8]{inputenc} % Determines encoding of the input. All input files have to use UTF8 encoding.

% Extended LaTeX functionality is enables by including packages with \usepackage{...}.
\usepackage{amsmath}    % Extended typesetting of mathematical expression.
\usepackage{amssymb}    % Provides a multitude of mathematical symbols.
\usepackage{mathtools}  % Further extensions of mathematical typesetting.
\usepackage{microtype}  % Small-scale typographic enhancements.
\usepackage[inline]{enumitem} % User control over the layout of lists (itemize, enumerate, description).
\usepackage{multirow}   % Allows table elements to span several rows.
\usepackage{booktabs}   % Improves the typesettings of tables.
\usepackage{subcaption} % Allows the use of subfigures and enables their referencing.
\usepackage[ruled,linesnumbered,algochapter]{algorithm2e} % Enables the writing of pseudo code.
\usepackage[usenames,dvipsnames,table]{xcolor} % Allows the definition and use of colors. This package has to be included before tikz.
\usepackage{nag}       % Issues warnings when best practices in writing LaTeX documents are violated.
\usepackage{todonotes} % Provides tooltip-like todo notes.
\usepackage{hyperref}  % Enables cross linking in the electronic document version. This package has to be included second to last.
\usepackage[acronym,toc]{glossaries} % Enables the generation of glossaries and lists fo acronyms. This package has to be included last.

% Define convenience functions to use the author name and the thesis title in the PDF document properties.
\newcommand{\authorname}{Lukas Baronyai} % The author name without titles.
\newcommand{\thesistitle}{Comparison and Implementation of LOD Frameworks} % The title of the thesis. The English version should be used, if it exists.
\newcommand{\thesissubtitle}{Finding evaluation and application of implicit existing best practice frameworks} % The subtitle of the thesis. The English version should be used, if it exists.

% Set PDF document properties
\hypersetup{
    pdfpagelayout   = TwoPageRight,           % How the document is shown in PDF viewers (optional).
    linkbordercolor = {Melon},                % The color of the borders of boxes around crosslinks (optional).
    pdfauthor       = {\authorname},          % The author's name in the document properties (optional).
    pdftitle        = {\thesistitle},         % The document's title in the document properties (optional).
    pdfsubject      = {Subject},              % The document's subject in the document properties (optional).
    pdfkeywords     = {a, list, of, keywords} % The document's keywords in the document properties (optional).
}

\setpnumwidth{2.5em}        % Avoid overfull hboxes in the table of contents (see memoir manual).
\setsecnumdepth{subsection} % Enumerate subsections.

\nonzeroparskip             % Create space between paragraphs (optional).
\setlength{\parindent}{0pt} % Remove paragraph identation (optional).

\makeindex      % Use an optional index.
\makeglossaries % Use an optional glossary.
%\glstocfalse   % Remove the glossaries from the table of contents.

% Set persons with 4 arguments:
%  {title before name}{name}{title after name}{gender}
%  where both titles are optional (i.e. can be given as empty brackets {}).
\setauthor{}{\authorname}{}{male}
\setadvisor{Pretitle}{Forename Surname}{Posttitle}{male}

% For bachelor and master theses:
\setfirstassistant{Pretitle}{Forename Surname}{Posttitle}{male}
\setsecondassistant{Pretitle}{Forename Surname}{Posttitle}{male}
\setthirdassistant{Pretitle}{Forename Surname}{Posttitle}{male}

% Required data.
\setaddress{Längenfeldgasse 28/8/5, 1120 Wien}
\setregnumber{1326526}
\setdate{\the\day}{\the\month}{\the\year} % Set date with 3 arguments: {day}{month}{year}.
\settitle{\thesistitle}{\thesistitle} % Sets English and German version of the title (both can be English or German).
\setsubtitle{\thesissubtitle}{\thesissubtitle} % Sets English and German version of the subtitle (both can be English or German).

% Select the thesis type: bachelor / master / doctor / phd-school.
% Bachelor:
\setthesis{bachelor}

% For bachelor and master:
\setcurriculum{Software and Information Engineering}{Software und Information Engineering} % Sets the English and German name of the curriculum.

\usepackage[backend=bibtex,bibencoding=ascii, natbib=true, defernumbers=true, style=ieee]{biblatex}
\addbibresource{bibliography.bib}

\begin{document}

\frontmatter % Switches to roman numbering.
% The structure of the thesis has to conform to
%  http://www.informatik.tuwien.ac.at/dekanat

\addtitlepage{naustrian} % German title page (not for dissertations at the PhD School).
\addtitlepage{english} % English title page.
\addstatementpage

%%%%%%%%%%%%%%%%%%%%%%%%%%%%%%%%%%%%%%%%%%%%%%%%%%%%%%%%%%%%%%

\begin{acknowledgements*}
\todo{Enter your text here.}
\end{acknowledgements*}
\begin{abstract}
\todo{Enter your text here.}
\end{abstract}

%%%%%%%%%%%%%%%%%%%%%%%%%%%%%%%%%%%%%%%%%%%%%%%%%%%%%%%%%%%%%%

% Select the language of the thesis, e.g., english or naustrian.
\selectlanguage{english}

% Add a table of contents (toc).
\setcounter{tocdepth}{2}
\tableofcontents % Starred version, i.e., \tableofcontents*, removes the self-entry.

% Switch to arabic numbering and start the enumeration of chapters in the table of content.
\mainmatter

%%%%%%%%%%%%%%%%%%%%%%%%%%%%%%%%%%%%%%%%%%%%%%%%%%%%%%%%%%%%%%

%some general acronyms
\newacronym{ld}{LD}{Linked Data}
\newacronym{od}{OD}{Open Data}
\newacronym{lod}{LOD}{Linked Open Data}

%%%%%%%%%%%%%%%%%%%%%%%%%%%%%%%%%%%%%%%%%%%%%%%%%%%%%%%%%%%%%%

\chapter{Introduction}

The Semantic Web is getting more and more popular and with it the need for ways to publish Linked (Open) Data. In order to cover these needs, a lot of different tools, frameworks and solutions came up, some covering the whole process of publishing L(O)D, others covering only steps in the process of it. These led to a vast diversity but also a big number of options. In order to develop a LOD project, the responsible stakeholders now have to choose among these many options.

\section{Research Question}

Aim of this paper is to compare existing and common LOD solutions to give responsible stakeholders at TU Wien a decision guidance for choosing one. The concrete research question is as following:

\textbf{RQ:} \textit{How do common LOD solutions~\footnotemark compare against each?}
\begin{enumerate}
\item \textbf{RQ1:} What are existing LOD solutions?
\item \textbf{RQ2:} What are criteria to compare solutions?
\item \textbf{RQ3:} How do they compare against each other?
\item \textbf{RQ4:} What can be a solution for TU Wien?
\end{enumerate}

The conducted work is a follow-up work of a previous study (see ~\cite{baronyai_publishing_2016}) done by the author and will build up on it.

\footnotetext{\textbf{Note}: The term \textit{solutions} is used here on purpose to abstract terms like \textit{framework}, \textit{tool} or \textit{all-in-one solution}. Using only the term \textit{framework} would lead to problems and cut out other options. For a more detailed discussion see section~\ref{def:framework}}.

\section{Methodology}

The work was done as a literature study. First a discussion of the term "framework" was done in order to achieve the research question. Then a range of existing L(O)D projects and application was investigated, to extract used technologies from them. From this, candidates were retrieved and classified. After excluding and filtering some of the candidates, they were compared in four criteria groups: Criteria from the above mentioned study, usability, data formats and the Linked Data Publishing Checklist.

\section{Structure of this Paper}
This paper starts with a state of the art section in which similar comparision will be described. The next chapter~\ref{chap:methodology} is the definition of the used methodology, where a discussion of the term "framework" will be done for this paper (section~\ref{def:framework}), the process of the literature study (section~\ref{meth_study}) and the classification (section~\ref{classification}) system will be defined.

In chapter~\ref{overview} the found solution will be reviewed and described as well as the excluded candidates (section~\ref{excluded}). In chapter~\ref{criteria} the criteria as well as their according scala will be defined. The final comparison will be done in chapter~\ref{comparison}, the summary of it can be found in section~\ref{comp_summary}.

The last research question, RQ4, will be answered in chapter~\ref{ch:tuwien}, investigating, which of the proposed solutions may be suitable for which situation at TU Wien.

The last chapter, \ref{ch:summary}, describes the overall summary and future work.
\chapter{State Of The Art}

\chapter{Methodology (RQ1)}\label{chap:methodology}
\section{About The Difficulty Of Comparing Solutions}\label{def:framework}

One could expect, that this thesis may compare "frameworks". But there comes problems with this term. What actually is a "framework"? Is the term well-defined enough to clearly refer to a solution?

\subsection{The Term "Framework"}

In order to use the term a first step must be to define \emph{what} a framework actually is, since it is a very generic term. One way to define it could be the definition by Roberts and Johnson,~\cite{roberts1996evolving}:

\begin{quotation}
Frameworks are reusable designs of all part of a software described by a set of abstract classes and the way instances of those collaborate
\end{quotation}

Another way could be the explanation by Riehle in his PhD thesis,~\cite{riehle2000framework}:

\begin{quotation}
Frameworks model a specific domain or an important aspect thereof. They represent the domain as an abstract design, consisting of abstract classes (or interfaces). The abstract design is more than a set of classes, because it defines how instances of the classes are allowed to collaborate with each other at runtime. Effectively, it acts as a skeleton, or a scaffolding, that determines how framework objects relate to each other.

A framework comes with reusable implementations in the form of abstract and concrete class implementations. Abstract implementations are abstract classes that implement parts of a framework abstraction (as expressed by an abstract class or interface), but leave crucial implementation decisions to subclasses. [...]
\end{quotation}

Both of them refer frameworks as tools for coding, used when writing own applications. One of the most classical examples might be the Spring Framework in the Java world. In the mentioned projects, Apache Jena and RDF4J mostly apply on this definition.

But the problem is here, that the term is not always used and understand in this way, LDIF and the Silk frameworks define themselves as such, but providing in facts a set of tools without necessarily needing coding to work with them (except configuration files). Others may see tools like the Information Workbench or D2RQ as a framework for publishing.

On a higher level, the architectures proposed in section~\ref{arch_frameworks} might be seen as a high-level or meta-framework. And since the proposed tools in the other sections (partially) are using these architectures, one could argue, that they therefore are also frameworks.

It can be seen, that is actually problematic to use this term in the context of this thesis since it is too broad and not well-defined. In order to solve this issue, the term "solution" will be used to abstract the term and referring either to "tool", "framework" or "all-in-one solution".

\subsection{Defining The Limits}

Next to the general problem about the term "framework", another problem is to set the borders of the examine topic. As mentioned in the introduction this thesis aims to compare LD solutions, the goal is to cover the whole process of publishing Linked Data, from the bottom persistence layer of accessing existing data, transforming data formats (e.g. relational to RDF), over cleaning and interlinking the data, over storing them in a triple store, up to making them available over an interface like SPARQL.

But there are not many tools/frameworks covering the whole process and supporting different data formats (e.g. relational data and CSV) at the same time. There are some tools like D2RQ only focusing on specific data formats, but providing the full stack, some tools like LDIF only focusing on a specific part of the process, without e.g. providing capabilities for SPARQL endpoints.

The best way is maybe using a stack of different tools to cover the whole workflow, combining them like Silk is integrated in LDIF. Or using the generic architecture, coding an own application and using partially the proposed tools.

But covering different areas, it is difficult to actually compare them. How to compare a persistence framework with a GUI framework? In order to handle this problem, a classification system will be introduced in section~\ref{classification} and the criteria introduced in section~\ref{criteria} will take this difficulty into account.

\section{Methodology Of The Literature Study}\label{meth_study}
In order to answer RQ1, a literature study was conducted, but since the scope is difficult to define as described in section~\ref{def:framework}, it was not a pure study. As the aim of this thesis is to compare \textit{common} frameworks and \textit{best practices}, it would be not sufficient to review every possible paper about a LD framework or tool, therefore another approach was chosen: deriving candidates from projects. In order to do that, the following process was used:

\begin{enumerate}
\item Identify \& find a LD application/project, ignoring the success of it
\item Find public documentation and/or scientific work of it
\item Analyse used technology, add as candidate if appropriate and if not disadvised
\item Classify candidates (see~\ref{classification})
\item Analyse reference work for possible input for 1.)
\item Analyse reference work of tool/framework at its documentation
\end{enumerate}

Using this approach led to a variety of candidates, which will be listed in section~\ref{overview}. The candidates from section~\ref{excluded} were mostly excluded because of step 2.), which ensured a better base for the following comparison.

\section{A Classification System}\label{classification}

\begin{table}[htb]
\centering
\begin{tabular}{|l|l|}
\hline
\textbf{Class}              & \textbf{Detail}                                                                                                                                                                                                                        \\ \hline
\textbf{Architecture}       & \begin{tabular}[c]{@{}l@{}}A general architecture without concrete technology. \\ A solution of this class can be used in combination \\ of any other class.\end{tabular}                                                        \\ \hline
\textbf{Full-Stack}         & \begin{tabular}[c]{@{}l@{}}A solution which covers the whole stack and \\ therefore does not need another component. \\ An "All-In-One" Solution\end{tabular}                                                                    \\ \hline
\textbf{Presentation layer} & \begin{tabular}[c]{@{}l@{}}A solution which only covers the presentation or \\ UI layer and therefore depends on other component.  \\ Managing how LD can be accessed from outside \\ and how the data are exposed.\end{tabular} \\ \hline
\textbf{Business Layer}     & \begin{tabular}[c]{@{}l@{}}A solution which only covers the business layer \\ and therefore depends on other components. \\ Managing how LD are processed.\end{tabular}                                                          \\ \hline
\textbf{Data Access Layer}  & \begin{tabular}[c]{@{}l@{}}A solution which only covers the data access layer \\ and therefore depends on other components. \\ Managing how LD are stored and accessed \\ by the application.\end{tabular}                          \\ \hline
\end{tabular}
\centering
\caption{Overview of the Classification}
\label{the_label}
\end{table}

Resulting from~\ref{def:framework} different classifications were introduced to find classification-based criteria and to balance out the vast variation of the results. The classifications can be seen in table~\ref{the_label}.

The Classifications are based on the idea, that a majority of applications are using in one way or another a variation or parts of the three layer architecture style, with components responsible for either UI, Business or Data Access. This does \textit{not} necessarily mean, that they use the full concepts of this architecture or even implementing this style. It is only assumed that a component have a responsibility mappable to one of the layers. Accordingly it is assumed, that a solution can be associated with one of these responsibilities.

It is arguable, if the differentiation between "Full-Stack" and labeling a solution with the three layer class is necessary. The additional "Full-Stack" class was added to emphasize the "All-In-One" approach of such a solution, meaning that all components are provided, no further components are need. This also means, that the included components of the different responsibilities are either harmonized to each other or do not differentiate between these responsibilities. On the other side labeling a tool with the three layer classes, does not implicit this and can also mean, that the support of each of this layer can be optional.
\chapter{Comparison (RQ2 \& RQ3)}
\todo{Enter your text here.}
\section{Comparison of the Frameworks}
\subsection{General architecture pattern}
Multitier/-layer Architecture
Typical:
\begin{enumerate}
\item Data Source/Input
\item Data Preparation
\item Data Storage (Triple Store)
\item Data Publication
\end{enumerate}
\subsection{Strategies}
\subsubsection{Data Preparation}
Extractors, RDF-Tranformers, Cleansing, Vocabulary Mapping
\subsubsection{Data Interlinking}
\subsubsection{Data Storage}
Triple Store, Relational Database, RDF Storage
\subsubsection{Data Publication}
SPARQL, CMS
\section{Comparison to All-In-One Solutions}
%\chapter{Implementation (RQ4)}
\todo{Enter your text here.}
\section{Domain (Publication Database)}
\section{Composed Architecture (Best practise)}
\subsection{Architecture Pattern}
\subsection{Strategies}
\section{Used Technologies}
\chapter{Critical reflection}
\todo{Enter your text here.}
\section{Existing Best Practice}
\section{Analysis of the Implementation}
\section{Applicability and Adaptability}
\chapter{Summary and future work}
\todo{Enter your text here.}

%%%%%%%%%%%%%%%%%%%%%%%%%%%%%%%%%%%%%%%%%%%%%%%%%%%%%%%%%%%%%%

\backmatter

%%%%%%%%%%%%%%%%%%%%%%%%%%%%%%%%%%%%%%%%%%%%%%%%%%%%%%%%%%%%%%

% Use an optional list of figures.
\listoffigures % Starred version, i.e., \listoffigures*, removes the toc entry.

% Use an optional list of tables.
%\cleardoublepage % Start list of tables on the next empty right hand page.
\listoftables % Starred version, i.e., \listoftables*, removes the toc entry.

% Use an optional list of alogrithms.
%\listofalgorithms
%\addcontentsline{toc}{chapter}{List of Algorithms}

% Add an index.
\printindex

% Add a glossary.
\printglossaries

%%%%%%%%%%%%%%%%%%%%%%%%%%%%%%%%%%%%%%%%%%%%%%%%%%%%%%%%%%%%%%

% Add a bibliography.
%\bibliographystyle{alpha}
%\bibliography{bibliography}

%\addcontentsline{toc}{chapter}{References}

\printbibliography[title={References to refereed scientific work}, keyword=refsw]

\printbibliography[title={References to non-refereed work}, keyword=nonrefsw]

\printbibliography[title={References to websites}, type=misc]

\end{document}