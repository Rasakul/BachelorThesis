% Copyright (C) 2014-2016 by Thomas Auzinger <thomas@auzinger.name>

\documentclass[draft,final]{vutinfth} % Remove option 'final' to obtain debug information.
%\documentclass[draft,final]{vutinfth} % Remove option 'final' to obtain debug information.

% Load packages to allow in- and output of non-ASCII characters.
\usepackage{lmodern}        % Use an extension of the original Computer Modern font to minimize the use of bitmapped letters.
\usepackage[T1]{fontenc}    % Determines font encoding of the output. Font packages have to be included before this line.
\usepackage[utf8]{inputenc} % Determines encoding of the input. All input files have to use UTF8 encoding.

% Extended LaTeX functionality is enables by including packages with \usepackage{...}.
\usepackage{amsmath}    % Extended typesetting of mathematical expression.
\usepackage{amssymb}    % Provides a multitude of mathematical symbols.
\usepackage{mathtools}  % Further extensions of mathematical typesetting.
\usepackage{microtype}  % Small-scale typographic enhancements.
\usepackage[inline]{enumitem} % User control over the layout of lists (itemize, enumerate, description).
\usepackage{multirow}   % Allows table elements to span several rows.
\usepackage{booktabs}   % Improves the typesettings of tables.
\usepackage{subcaption} % Allows the use of subfigures and enables their referencing.
\usepackage[labelfont=bf]{caption}
\usepackage[ruled,linesnumbered,algochapter]{algorithm2e} % Enables the writing of pseudo code.
\usepackage[usenames,dvipsnames,table]{xcolor} % Allows the definition and use of colors. This package has to be included before tikz.
\usepackage{nag}       % Issues warnings when best practices in writing LaTeX documents are violated.
\usepackage{todonotes} % Provides tooltip-like todo notes.
\usepackage{hyperref}  % Enables cross linking in the electronic document version. This package has to be included second to last.
\usepackage[acronym,toc]{glossaries} % Enables the generation of glossaries and lists fo acronyms. This package has to be included last.

\usepackage{graphicx,rotating}
\usepackage{csquotes}

% Define convenience functions to use the author name and the thesis title in the PDF document properties.
\newcommand{\authorname}{Lukas Baronyai} % The author name without titles.
\newcommand{\thesistitle}{Comparison of LOD Solutions} % The title of the thesis. The English version should be used, if it exists.
\newcommand{\thesissubtitle}{Finding evaluation of existing solutions and giving a guide to use at TU Wien} % The subtitle of the thesis. The English version should be used, if it exists.

% Set PDF document properties
\hypersetup{
    pdfpagelayout   = TwoPageRight,           % How the document is shown in PDF viewers (optional).
    linkbordercolor = {Melon},                % The color of the borders of boxes around crosslinks (optional).
    pdfauthor       = {\authorname},          % The author's name in the document properties (optional).
    pdftitle        = {\thesistitle},         % The document's title in the document properties (optional).
    pdfsubject      = {Comparison of LOD Frameworks},              % The document's subject in the document properties (optional).
    pdfkeywords     = {LOD, linked data, linked open data, comparison, framework, LDIF, Apache, Jena, Euclid, LUCERO, Linked Data book, D2RQ, Information Workbench, Eclipse, RDF4J} % The document's keywords in the document properties (optional).
}

\setpnumwidth{2.5em}        % Avoid overfull hboxes in the table of contents (see memoir manual).
\setsecnumdepth{subsection} % Enumerate subsections.

\nonzeroparskip             % Create space between paragraphs (optional).
\setlength{\parindent}{0pt} % Remove paragraph identation (optional).

%\makeindex      % Use an optional index.
%\makeglossaries % Use an optional glossary.
%\glstocfalse   % Remove the glossaries from the table of contents.

% Set persons with 4 arguments:
%  {title before name}{name}{title after name}{gender}
%  where both titles are optional (i.e. can be given as empty brackets {}).
\setauthor{}{\authorname}{}{male}
\setadvisor{Prof. Dr.}{Stefan Biffl}{}{male}

% For bachelor and master theses:
\setfirstassistant{MSc., PhD}{Marta Sabou}{}{female}

% Required data.
\setaddress{Huttengasse 51/10, 1160 Wien}
\setregnumber{01326526}
\setdate{\the\day}{\the\month}{\the\year} % Set date with 3 arguments: {day}{month}{year}.
\settitle{\thesistitle}{\thesistitle} % Sets English and German version of the title (both can be English or German).
\setsubtitle{\thesissubtitle}{\thesissubtitle} % Sets English and German version of the subtitle (both can be English or German).

% Select the thesis type: bachelor / master / doctor / phd-school.
% Bachelor:
\setthesis{bachelor}

% For bachelor and master:
\setcurriculum{Software and Information Engineering}{Software und Information Engineering} % Sets the English and German name of the curriculum.

\usepackage[backend=bibtex,bibencoding=ascii, natbib=true, defernumbers=true, style=ieee]{biblatex}
\addbibresource{bibliography.bib}

\begin{document}

\frontmatter % Switches to roman numbering.
% The structure of the thesis has to conform to
%  http://www.informatik.tuwien.ac.at/dekanat

\addtitlepage{naustrian} % German title page (not for dissertations at the PhD School).
\addtitlepage{english} % English title page.
\addstatementpage

%%%%%%%%%%%%%%%%%%%%%%%%%%%%%%%%%%%%%%%%%%%%%%%%%%%%%%%%%%%%%%

%\begin{danksagung*}
\todo{Ihr Text hier.}
\end{danksagung*}

\begin{acknowledgements*}
\todo{Enter your text here.}
\end{acknowledgements*}
\begin{abstract}

Along with the rising popularity of the Semantic Web and Linked Open Data, there are many tools and frameworks present which supporting building LOD applications. It is hard to choose an approach among the available options, which is especially true for the TU Wien. This paper aims to give an overview of different kind of LOD tools and frameworks and compare them among each other to give stakeholder at TU Wien a guideline for a future LOD project at the university. In order to do that, a literature study was conducted to choose candidates for the comparison. These 8 candidates (Euclid project, LUCERO, Linked Data book, D2RQ Platform, Information Workbench, LDIF, Eclipse RDF4J and Apache Jena) were then compared with a new comparison framework, developed for this paper. Problems arised due the different nature of the investigated candidates (frameworks VS tools). At the end this thesis, the results are analysed for a usage at TU Wien.

\end{abstract}

%%%%%%%%%%%%%%%%%%%%%%%%%%%%%%%%%%%%%%%%%%%%%%%%%%%%%%%%%%%%%%

% Select the language of the thesis, e.g., english or naustrian.
\selectlanguage{english}

% Add a table of contents (toc).
\setcounter{tocdepth}{2}
\tableofcontents % Starred version, i.e., \tableofcontents*, removes the self-entry.

% Switch to arabic numbering and start the enumeration of chapters in the table of content.
\mainmatter

%%%%%%%%%%%%%%%%%%%%%%%%%%%%%%%%%%%%%%%%%%%%%%%%%%%%%%%%%%%%%%

%some general acronyms
\newacronym{ld}{LD}{Linked Data}
\newacronym{od}{OD}{Open Data}
\newacronym{lod}{LOD}{Linked Open Data}

%%%%%%%%%%%%%%%%%%%%%%%%%%%%%%%%%%%%%%%%%%%%%%%%%%%%%%%%%%%%%%

\chapter{Introduction}

The Semantic Web is getting more and more popular and with it the need for ways to publish Linked (Open) Data. In order to cover these needs, a lot of different tools, frameworks and solutions came up, some covering the whole process of publishing L(O)D, others covering only steps in the process of it. These led to a vast diversity but also a big number of options. In order to develop a LOD project, the responsible stakeholders now have to choose among these many options.

\section{Research Question}

Aim of this paper is to compare existing and common LOD solutions to give responsible stakeholders at TU Wien a decision guidance for choosing one. The concrete research question is as following:

\textbf{RQ:} \textit{How do common LOD solutions~\footnotemark compare against each?}
\begin{enumerate}
\item \textbf{RQ1:} What are existing LOD solutions?
\item \textbf{RQ2:} What are criteria to compare solutions?
\item \textbf{RQ3:} How do they compare against each other?
\item \textbf{RQ4:} What can be a solution for TU Wien?
\end{enumerate}

The conducted work is a follow-up work of a previous study (see ~\cite{baronyai_publishing_2016}) done by the author and will build up on it.

\footnotetext{\textbf{Note}: The term \textit{solutions} is used here on purpose to abstract terms like \textit{framework}, \textit{tool} or \textit{all-in-one solution}. Using only the term \textit{framework} would lead to problems and cut out other options. For a more detailed discussion see section~\ref{def:framework}}.

\section{Methodology}

The work was done as a literature study. First a discussion of the term "framework" was done in order to achieve the research question. Then a range of existing L(O)D projects and application was investigated, to extract used technologies from them. From this, candidates were retrieved and classified. After excluding and filtering some of the candidates, they were compared in four criteria groups: Criteria from the above mentioned study, usability, data formats and the Linked Data Publishing Checklist.

\section{Structure of this Paper}
This paper starts with a state of the art section in which similar comparision will be described. The next chapter~\ref{chap:methodology} is the definition of the used methodology, where a discussion of the term "framework" will be done for this paper (section~\ref{def:framework}), the process of the literature study (section~\ref{meth_study}) and the classification (section~\ref{classification}) system will be defined.

In chapter~\ref{overview} the found solution will be reviewed and described as well as the excluded candidates (section~\ref{excluded}). In chapter~\ref{criteria} the criteria as well as their according scala will be defined. The final comparison will be done in chapter~\ref{comparison}, the summary of it can be found in section~\ref{comp_summary}.

The last research question, RQ4, will be answered in chapter~\ref{ch:tuwien}, investigating, which of the proposed solutions may be suitable for which situation at TU Wien.

The last chapter, \ref{ch:summary}, describes the overall summary and future work.
\chapter{State of the art (RQ1)}
\todo{Enter your text here.}
\section{Frameworks}
\subsection{Euclid Project}
\url{http://www.euclid-project.eu/modules/chapter5}
\subsection{LUCERO}
\url{https://code.google.com/archive/p/luceroproject/wikis/StepByStepDocumentation.wiki}
\subsection{LD-Patterns}
\url{http://patterns.dataincubator.org/book/linked-data-patterns.pdf }
\subsection{Linked Data: Evolving the Web into a Global Data Space (Heath, Bizer)}
\url{http://linkeddatabook.com/editions/1.0/#htoc61}
\section{All-In-One Solutions}
\subsection{D2R Server}
\url{http://d2rq.org/d2r-server}
\section{Excluded Tools}
\subsection{LOD2 Stack}
(too general, only stack of technologies)
\url{http://stack.linkeddata.org/lod2/}
\subsection{LODUM}
(no public available framework)
\url{http://lodum.de/}
\chapter{Methodology (RQ2 \& RQ3)}
\todo{Enter your text here.}
\section{Definitions for this paper}
\subsection{Framework}~\label{def:framework}
\section{About the difficulty of comparing frameworks}
\section{Criteria}

\begin{itemize}
\item Maintainability
How much effort needs the maintenance? 

\item Data quality 
\begin{itemize}
\item Data freshness (ability to handle new data)
\item Flexibility (of ontology) (deal with heterogenous and/or legacy data)
\end{itemize}

\item Usability: (adopted from Abran et.al.~\cite{abran2003usability} to fit)
\begin{itemize}
\item Effectiveness (How well do the users achieve their goals using the system?)
\item Efficiency (Time to achieve one task, complexity to handle)
\item Satisfaction
\item Security
\item Learnability (Documentation)
\item Performance
\end{itemize}

\item Available data formats (HTML, Relational Databases, Wrapping Existing Application or Web APIs, XML, Tables/Spreadsheets)

%LD book
\item Linked Data Publishing Checklist (from Heath et.al.~\cite{heath2011linked})
\begin{itemize}
\item Does your data set links to other data sets?
\item Do you provide provenance metadata?
\item Do you provide licensing metadata?
\item Do you use terms from widely deployed vocabularies?
\item Are the URIs of proprietary vocabulary terms dereferenceable?
\item Do you map proprietary vocabulary terms to other vocabularies?
\item Do you provide data set-level metadata?
\item Do you refer to additional access methods?
\end{itemize}
\end{itemize}
\chapter{Overview of solutions (RQ1)}\label{overview}
In order to compare solutions an understanding of existing solutions is necessary. This section will look at existing solutions, what kind of solutions they are, which of them can be used for this paper and which must be excluded. Furthermore, this section aims to understand how solutions look like and will examine the architecture of them.

\section{Architectures Of Frameworks}\label{arch_frameworks}
In this subsection the paper will look into three proposed models how solutions (and/or implemented LD-applications) should look like. There are many other existing architectures and ongoing projects exposing data as Linked (Open) Data, this paper will use the following as representation of them.

\newacronym{euclid}{EUCLID}{EdUcational Curriculum for the usage of Linked Data}
\subsection{Euclid Project}\index{Architectures!Euclid Project}

\begin{figure}[h]
	\centering
\includegraphics[width=0.5\textwidth]{img/euclid_logo.png}
\end{figure}

\begin{figure}[htbp]
	\centering
\includegraphics[width=.8\textwidth]{img/euclid_architecture.png}
	\caption{General EUCLID architecture}
	\label{euclid_architecture}
\end{figure}

The EUCLID project~\footnote{~\cite{euclid:home}} (EdUcational Curriculum for the usage of Linked Data) was founded under the \emph{Seventh Framework Programme of Research and Technological Development}, a funding program of the European Union/European Commission for 2007-2013~\footnote{~\cite{eu:fp7}}\footnote{ EUCLID in the CORDIS database: \url{http://cordis.europa.eu/project/rcn/103709_en.html}}.

Aim of the project was (and still is) to gather existing knowledge and expertise of \emph{"researchers, technology enthusiasts and early adopters in various European Member States"} and provide that accumulated as educational resources to enable the  full benefit of L(O)D for European businesses. The project built upon a consortium experienced in \emph{"over 20 LD projects with over 40 companies and public offices in more than 10 countries"}~\cite{euclid:about}

The outcome of this project is a a range of learning materials, fragmented into modules, and eLearning distribution channels. Overall there are six modules:

\begin{enumerate}

	\item \textbf{Introduction and Application Scenarios}
	The introduction provides the knowledge to understand, \emph{what} Linked Data are, the main principles, the standards and the required technologies. Further, an overview how to publish and to consume the data is given.
	
	\item \textbf{Querying Linked Data}
	This chapter mainly describes SPARQL and how to use it for querying and updating.
	
	\item \textbf{Providing Linked Data}
	This module deals with the production and exposure of Linked data, using the tools as R2RML (for relational databases), Open Refine (for spreadsheets), GATECloud (for natural language) and Silk (for interlinkage between datasets, see section~\ref{silk} for details about this tool).
	
	\item \textbf{Interaction with Linked Data}
	The projects describes in this chapter, how to explore Linked Data, using visualization tools, semantic browsers and applications, introducing search options like faceted search, concept-based search and hybrid search.
	
	\item \textbf{Creating Linked Data Applications}
	This module describes how to build a Linked Data Application, which technologies to use and how to integrate common Web APIs.
	
	\item \textbf{Scaling up}
	Finally this chapter examines the main issues of scalability regarding Linked Open Data and describes the relationship to Big Data.
	
\end{enumerate}

For this paper module 3 and 5~\footnote{~\cite{euclid:chap5}} are the most interesting. Module 3 describes some useful technologies for various steps on the way of exposing L(O)D, but module 5 introduce a high level architecture and some patterns, how a L(O)D application might look like (see ~\cite{euclid:chap5} for details). In detail, they provide a three-tier architecture (see figure~\ref{euclid_architecture} and three architecture patterns.

The architecture is very generic and consists of the classic three tiers: presentation, logic and data, each independent to the overlaying tier. Since the presentation and logic layer does not concern the actual publishing of the data, the data layer is the interesting one here. The layer consists of the \emph{Data Access Component}, which represents the access to different data types like relational data or other Web APIs and transforms the data to RDF, the \emph{Data Integration Component}, which does the vocabulary mapping and interlinking for the cleansing in order to e.g. identify and fix ambiguities in resource names, and finally the \emph{Triple Store}, holding the integrated dataset for exposing it to the web.

The mentioned patterns to use for implementations are:

\begin{itemize}

\item \textbf{Crawling pattern}
Used for loading the data in advance and storing them in a triple store, increasing the efficiency of data access. In exchange, the data might not be up to date when accessed.

\item \textbf{On-The-Fly Dereferencing Pattern}
Meaning that the URIs are dereferenced when the application need to access the data. This pattern provides up to date data but for the cost of performance when dereferencing many URIs.

\item \textbf{(Federated) Query Pattern}
Describing the use of complex queries on a fix set of data sources, enabling to work with current data directly retrieved from the sources. The pattern offers an access up-to-date data with adequate response time in specific situations but for the cost of the complex problem to find optimal queries.
\end{itemize}

\newacronym{lucero}{LUCERO}{Linking University Content for Education and Research Online}
\subsection{LUCERO}\index{Architectures!LUCERO}

%\begin{figure}[htbp]
%	\centering
%\includegraphics[width=0.15\textwidth]{img/lucero_logo.png}
%\end{figure}

\begin{figure}[htbp]
	\centering
\includegraphics[width=\textwidth]{img/lucero_architecture.png}
	\caption{LUCERO work flow \& architecture}
	\label{lucero_architecture}
\end{figure}

The LUCERO project ("Linking University Content for Education and Research Online")~\footnote{The code is available in the Google Code Archive: \url{https://code.google.com/archive/p/luceroproject/wikis/StepByStepDocumentation.wiki}} was a project at the Open University, aiming to \emph{"scope, prototype, pilot and evaluate reusable, cost-effective solutions relying on the linked dataprinciples and technologies for exposing and connecting educational and research content"}. It was founded for one year by the JISC Information Environment 2011 Programme under the call Deposit of research outputs and Exposing digital content for education and research.~\cite{lucero:about}

The projects connected with other organizations through LinkedUniversities.org~\footnote{\url{http://linkeduniversities.org/}} to gather common issues and practices. The outcome was the first university linked data platform,~\url{http://data.open.ac.uk/}, with a lot of impact on The Open University and the education community.

Looking at the architecture in figure~\ref{lucero_architecture} comparing to the Euclid architecture seen in the previous section, there are quite a lot of similarities. Both have components for accessing different kinds of data, here called \emph{Extractors}, for cleaning the data, here called \emph{Cleaner}, and a Triple Store, holding the data available. The lanes "Collect", "Extract", "Link" and "Store" can be seen as the data layer from the classic three-tier architecture, the "Expose" lane as the logic and presentation layer. 

Both using the crawling pattern to extract, map and store the data in a Linked Data format instead of transforming them for every request.

\subsubsection{TABLOID}\index{Tools!TABLOID}
One of the outcomes next to the LOD application itself was the Tabloid ("Toolkit ABout Linked Open Institutional Data"), \emph{"a toolkit intended to help institutions and developers to both publish and consume linked data"}. It contains work-flows, documentations, examples and tools~\cite{lucero:tabloid} trying to address different roles such as managers, developers and users. Tabloid tries to help people to understand LD, what can be done with it and gives advice on a technical perspective, how to publish and consume LD, providing at the same time a detailed and generic way.

\subsection{Linked Data book}\index{Architectures!Linked Data Book}

\begin{figure}[htbp]
	\centering
\includegraphics[width=\textwidth]{img/ld_architecture.png}
	\caption{Linked Data Publishing Options and Workflows according to the LD book}
	\label{ld_architecture}
\end{figure}

Another big effort among many others of describing LD in general, how to publish and consume them and how to implement applications, was done by the book \emph{"Linked Data: Evolving the Web into a Global Data Space"} by Heath and Bizer~\cite{heath2011linked}, which received a lot of attention.

The book aims in general to give a basic understanding of LD and describing publication and consumption of LD. They providing advices and best practices, including architectures approaches, identifying the right set of URIs and vocabulary and much more. They also described an architecture, to be seen in figure~\ref{ld_architecture}

Next to patterns they also provide a general workflow for LD publishing, see figure~\ref{ld_architecture}. But comparing to the introduced architectures in the previous sections, the workflow has a different approach: instead of holding the data in a Triple Store, the workflow access and transforms the raw data on-the-fly for every request.

Next to this workflow, the book also provides various "recipes" for publishing LD and one of them is also to hold the data in a triple store as shown by Euclid and LUCERO. Furthermore the book provides a guide for the D2R-Server, which will be described in section~\ref{d2rq}.

\newpage
\section{Frameworks}

\subsection{D2RQ Platform}\label{d2rq}\index{Framework!D2RQ Platform}

\begin{figure}[htbp]
	\centering
\includegraphics[width=0.8\textwidth]{img/d2rq_architecture.png}
	\caption{D2R Server architecture}
	\label{d2rq_architecture}
\end{figure}

\begin{verse}
\textbf{NOTE:} The last update on the D2RQ platform was in 2012 (version 0.8.1) and on the D2R Server in 2009 (version 0.7)
\end{verse}

The D2RQ platform~\footnote{\url{http://d2rq.org}} was introduced by the Free University of Berlin and provides a database-to-RDF mapping. It is licensed under the terms of the GNU General Public License. 

To map a relational database the platform provides a declarative mapping language, expressed in RDF, which is then be used to provide access to the database in the following, read-only, ways:~\cite{d2rq_w3c}

\begin{itemize}
\item \textbf{RDF dumps}
\item \textbf{RDF APIs}
\item \textbf{SPARQL endpoint} (D2R Server)
\item \textbf{Linked Data}
\item \textbf{HTML view} (D2R Server)
\end{itemize}

For an overview of the framework structure see figure~\ref{d2rq_architecture}.

\subsubsection{D2R Server}\index{Framework!D2RQ Platform!D2R Server}
Part of the platform is the D2R Server~\footnote{\url{http://d2rq.org/d2r-server}}, which provides the public access to the platform over SPARQL and HTML, publishing it to the semantic web. More concrete, the server provides a dereferencing interface, for HTTP request dereferencing, and a SPARQL interface. 

The server uses the mentioned \textbf{On-The-Fly Dereferencing Pattern} and does not provide a triple store, therefore it may be not has as good performance than tools with a triple store, although the team made a great effort to improve it.

Part of the server is also a tool which generates automatically a corresponding mapping and RDF vocabulary for an existing table structure, using table names as class names and column names as property names. The generated mapping file can then be customised.~\cite{bizer2006d2r}

The following applications are examples using D2R-Server:

\begin{itemize}
\itemsep0pt
\item DBLP Bibliography (University of Hannover)~\footnote{\url{http://dblp.uni-trier.de/}}
\item DBtune (University of London)~\footnote{\url{http://dbtune.org/}}
\item Database of the Nobel Prize~\footnote{\url{http://data.nobelprize.org/}}
\end{itemize}

\subsection{Information Workbench}\index{Framework!Information Workbench}

\begin{figure}[htbp]
	\centering
\includegraphics[width=0.9\textwidth]{img/information_workbench_architecture.png}
	\caption{Architecture of the Information Workbench}
	\label{iw_architecture}
\end{figure}

The Information Workbench~\footnote{\url{https://www.fluidops.com/en/products/information_workbench/}} is a high customisable tool to support the building of Linked Data applications, from basic data integration up to rich UI and visualisations. The tool is developed by fluidOps and is published as Community Edition free available and under an Open Source License with a limited selection of capablities and only for non-productive use (educational use, testing, development). The enterprise edition is also available but not for free.

\newpage

The workbench consists of four layers (see figure~\ref{iw_architecture} for an overview):~\cite{haase2011information}~\cite{gossenainformation}

\begin{itemize}

\item \textbf{Persistence}
Using so-called \emph{providers}, the layers offer capabilities to integrate and convert data from different data source and stores them in a central triple store. Alternatively it also supports virtualised integration of local and public Linked Data sources using a \emph{federation layer}.

\item \textbf{Platform}
On top of the persistence layer the core Platform layer a selection of modules and functionalities covering generic needs of Linked Data applications, the most important are a \emph{Semantic Wiki \& Widget Engin}e, an \emph{User Management \& Access Control}, a \emph{Search \& Analytics Engine} and a \emph{Workflow Engine}.

\item \textbf{SDK} To support customised applications the workbench provides a SDK (Solution Development Kit) for developers to build domain specific applications, including \emph{extensible data providers}, \emph{data management facilities}, modified \emph{ontologies}, \emph{templates}, \emph{widgets} and different APIs for extensive \emph{system configuration}, \emph{rules} and \emph{workflows}.

\item \textbf{Solution}
On top of all layer stands the final solution, the application itself, which is either directly deployed through a RESTful API or over a zipped file for other installation approaches.

\end{itemize}

The resulting application is again customisable by widgets and different views, enabling data exploration and visualisation.

\subsection{LDIF – Linked Data Integration Framework}\index{Framework!LDIF}\newacronym{ldif}{LDIF}{Linked Data Integration Framework}

\begin{figure}[htbp]
	\centering
\includegraphics[width=\textwidth]{img/ldif_context.png}
	\caption{LDIF in the context of a LD application}
	\label{ldif_context}
\end{figure}

\begin{figure}[htbp]
	\centering
\includegraphics[width=0.8\textwidth]{img/ldif_components.png}
	\caption{Components of LDIF}
	\label{ldif_components}
\end{figure}


LDIF~\footnote{\url{http://ldif.wbsg.de/}} was developed by the University of Mannheim and is published under the terms of the Apache Software License. It is implemented in Scala and aims to translate \emph{"heterogeneous Linked Data from the Web into a clean, local target representation while keeping track of data provenance."}.

From a component perspective, LDIF consists of pluggable modules and a runtime environment, managing the data flows between them. The modules are:~\cite{schultz2011ldif}~\cite{schultz2012ldif}

\begin{itemize}

\item \textbf{Data Access Modules \& Scheduler}
For accessing the data to transform, LDIF provides several ways to import them. These import jobs are managed by a scheduler, which frequently fills a local cache. The module supports Triple/Quade Dump (for RDF/XML, N-Triples, N-Quads and Turtle formats), Crawler (using LDSpider~\footnote{\url{https://github.com/ldspider/ldspider}}) and SPARQL imports.

\item \textbf{Data Translation}
For translating Web data using different vocabularies into a single target vocabulary, LDIF uses the R2R Mapping Language~\footnote{\url{http://wifo5-03.informatik.uni-mannheim.de/bizer/r2r/spec/}}.

\item \textbf{Identity Resolution}
To find different URIs in different data pointing to the same entity, LDIF employs the Silk Link Discovery Framework with the Silk - Link Specification Language (Silk-LSL).

\item \textbf{Data Quality Assessment and Fusion}
For quality assessment, LDIF uses the Sieve Data Quality Assessment and Data Fusion Framework~\footnote{\url{http://wifo5-03.informatik.uni-mannheim.de/bizer/sieve/}}.

\item \textbf{Data Output}
In the final step, LDIF write the cleaned data together with the provenance information in a single N-Quads file or without the meta-information in a N-Triples file.
\item \textbf{Runtime Environment} 
As mentioned, the runtime environment manage the data flow between each module, providing an in-memory (fast, but limited scalable), a RDF store (using Apache Jena TDB and SPARQL queries, better scalable for the price of performance) and an Hadoop version.

\end{itemize}

\subsubsection{Silk}\index{Framework!Silk}\label{silk}
Silk~\footnote{\url{http://silkframework.org/}} is \emph{"an open source framework for integrating heterogeneous data sources."} using the declarative Silk - Link Specification Language (Silk-LSL). It generates RDF links between data sets by custom link specifications. There are three different variations:~\cite{isele2010silk}

\begin{itemize}
\item \textbf{Silk Single Machine} generates RDF links between two data items on a single machine.
\item \textbf{Silk MapReduce} is for big scale datasets, using Hadoop and distributes to multiple machines.
\item \textbf{Silk Server} is intended be used as an identity resolution component of as Linked Data consuming application. It provides a REST interface and runs as an HTTP server.
\end{itemize}

For details about the Link Discovery Framework see~\cite{volz2009silk} and~\cite{jentzsch2010silk}, for the server version consult~\cite{isele2010silk}.

\subsection{Eclipse RDF4J (formerly Sesame)}
Eclipse RDF4J~\footnote{\url{http://rdf4j.org/}} (formerly known as Sesame) is \emph{a powerful Java framework for processing and handling RDF data. This includes creating, parsing, scalable storage, reasoning and querying with RDF and Linked Data. It offers an easy-to-use API that can be connected to all leading RDF database solutions.} It can be used as an embedded part of an application or as a stand-alone server.

Originally developed as Sesame by Aduna as part of the "On-To-Knowledge" project (1999-2002), it was official forked into RDF4J. It is licensed under a BSD-style license.

The framework comes with many components, like Alibaba, an API for mapping Java classes onto ontologies. The RDF database API is unlikely similar solutions, it consists of stackable interfaces for adding functionality. Next to the intern abstract storage engine (SAIL, Storage and Inference Layer), many other triplestores are supported, like  Ontotext GraphDB, Mulgara, and AllegroGraph

\subsection{Apache Jena}

Apache Jena~\footnote{\url{https://jena.apache.org/}} is \emph{a free and open source Java framework for building Semantic Web and Linked Data applications}. It was originally developed by HP Laboratories and now maintained by the Apache Software Foundation and is licensed under the Apache License 2.0.

The framework provides an API to extract data from and write to RDF, supporting relational databases, RDF/XML, Turtle and Notation 3. In contrast to RDF4J it also supports OWL.

More concrete, Jena can be used to manipulate RDF data, storing them in a triple store and publish it as a SPARQL access point. This HTTP interface is called \emph{Fuseki}, which is in fact a sub-project of Jena and can be also run as stand-alone server using the Jetty web server.

\newpage
\section{Excluded Tools and Projects}\label{excluded}

\subsection{LD-Patterns}\index{Framework!LD-Patterns}
The Linked Data Patterns book by Dodds and Davis (see~\cite{dodds2011linked}) tried to give an overview of existing design pattern regarding LD. But they don't give concrete architectures or architecture relating informations, so this paper will not use its content. But it is suggested, that this design pattern catalogue is used additionally when creating an application.

\subsection{LOD2 Stack}\index{Tools!LOD2 Stack}

The LOD2 stack, introduced by Auer et. al., is \emph{is an integrated distribution of aligned tools which support the whole life cycle of Linked Data from extraction, authoring/creation via enrichment, interlinking, fusing to maintenance.}~\cite{auer2012managing} For this paper the proposed stack of technology was too generic to compare it with other frameworks and the website of the project~\footnote{\url{http://stack.linkeddata.org/lod2//}} was at point of writing this paper offline, therefore it was excluded of this paper.

\subsection{LODUM}\index{Other LOD Projects!LODUM}

Another interesting project is the LODUM project (Linked Open Data University of Münster), the Open Data initiative of the university, hosted at the Institute for Geoinformatics' Semantic Interoperability Lab (MUSIL). The project team has co-initiated both LinkedUniversities.org and LinkedScience.org.

It was excluded for this paper because the project don't provide public documentation of their architecture or any other part of their technical details
\url{http://lodum.de/}

\subsection{Synth and SHDM}\index{Framework!Synth}

\begin{figure}[htbp]
	\centering
\includegraphics[width=\textwidth]{img/synth-concept.png}
	\caption{Concept of the Synth Architecture}
	\label{synth_concept}
\end{figure}

Synth~\footnote{\url{http://www.tecweb.inf.puc-rio.br/synth}} is a development environment for building SHDM~\footnote{\url{https://www.w3.org/2005/Incubator/model-based-ui/wiki/SHDM_-_Semantic_Hypermedia_Design_Method}} (Semantic Hypermedia Design Method) modelled applications, providing a set of modules, receiving SHDM generated models. Synth comes with a web browser GUI for adding and editing these models. A conceptual view of the architecture can be seen in figure~\ref{synth_concept}, where the dashed boxes are modules and the whites boxes insides the module components.~\cite{desynth} The authors de Souza Bomfim and Schwabe describe in two papers, how a Linked Data application can be build with the environment: ~\cite{desynth} and~\cite{de2011design}.

Since their description is very abstract and there are no further documentations of the tool, it was excluded for this paper. 
\chapter{Criteria (RQ2)}\label{criteria}

Since there are a wide variety of solutions, it is hard to find a set of criteria which can be applied to all in the same way in order to compare them. Therefore this paper will use 4 different criteria groups to ensure higher cover of them and allowing at the same time, that criteria might not be applicable for some solutions.

\section{Criteria Group 1: Criteria from Previous Study}

\begin{table}[htb]
\centering
\begin{tabular}{|l|c|l|}
\rowcolor[HTML]{EFEFEF} 
\hline
\textbf{Criteria} & \textbf{Scala} & \textbf{Explanation}                                 \\ \hline
Maintainability   & +/-            & \begin{tabular}[c]{@{}l@{}}How much effort needs the maintenance?  \\ (less/much)\end{tabular} \\ \hline
Data Freshness    & yes/ no        & Can it deal with new data?                        \\ \hline
Flexibility       & yes/ no        & \begin{tabular}[c]{@{}l@{}}Can it deal with heterogenous and/or legacy data? \\ Can it deal with changes in the ontology?\end{tabular} \\ \hline
\end{tabular}
\caption{Criteria group 1}
\label{tbl_cr_1_scala}
\end{table}

From a previous study~\cite{baronyai_publishing_2016} conducted by the author, users at TU Wien are expecting and requesting from a LOD application: \textbf{clear data ownership}, management of \textbf{data freshness} and \textbf{data quality} and \textbf{maintenance}. Since data ownership is a concern of organisation and cannot be clarified by a tool, data freshness, data quality and maintainability are introduced as criteria. Since data quality is a very generic term, it will be used as criteria category. Additionally a concern of the stakeholders from the paper was how to deal with legacy data, therefore flexibility is included to Data quality.

The resulting criteria can be seen in table~\ref{tbl_cr_1_scala}.

\section{Criteria Group 2: Usability}

\begin{table}[htb]
\centering
\begin{tabular}{|l|c|l|}
\hline
\rowcolor[HTML]{EFEFEF} 
\textbf{Criteria} & \textbf{Scala} & \textbf{Explanation}                                                                                                \\ \hline
Effectiveness     & +/-            & \begin{tabular}[c]{@{}l@{}}How well do the users achieve their goals using the system? \\ (good/bad)\end{tabular}   \\ \hline
Efficiency        & +/-            & \begin{tabular}[c]{@{}l@{}}What resources are consumed in order to achieve their goals? \\ (less/much)\end{tabular} \\ \hline
Satisfaction      & +/-            & \begin{tabular}[c]{@{}l@{}}How do the users feel about their use of the system? \\ (good/bad)\end{tabular}          \\ \hline
Security          & +/-            & \begin{tabular}[c]{@{}l@{}}How well is security ensured? \\ (good/bad)\end{tabular}                                 \\ \hline
Learnability      & +/-            & \begin{tabular}[c]{@{}l@{}}How much time is needed to learn the system? \\ (less/much)\end{tabular}                 \\ \hline
\end{tabular}
\caption{Criteria Group 2: Usability}
\label{tbl_cr_2_scala}
\end{table}

In every software application and especially solutions designed for end-users, usability is a huge and very important point these days. There are many definitions and measurements, ISO and other proposed models, trying to classify and define usability. This paper will use an enhanced ISO model, proposed by Abran et.al.~\cite{abran2003usability}. Since the goal of this paper is not a complete usability analysis of the tools/frameworks and the variety of the chosen tools is too wide, the analysis of these aspect will be more of a general type. Abran et.al. are proposing various measurements for the different categories of their model, the interested reader might use them for a detailed analysis. For this paper, these measurements will only be used as a guideline to estimate an assessment for the tools.

The model can be seen in table~\ref{tbl_cr_2_scala}

\section{Criteria Group 3: Data Formats}

An important aspect for the final decision for one of the tools can be the supported data format. It might be an (external) requirement or resulting from the fact of existing data. Since this highly depends on the context and the use case, this criteria will not be rated in any way, this paper will only line out the supported data formats.

\section{Criteria Group 4: Linked Data Publishing Checklist}

\begin{table}[htb]
\centering
\begin{tabular}{|l|}
\rowcolor[HTML]{EFEFEF} 
\hline
\multicolumn{1}{|c|}{\textbf{Criteria}}                        \\ \hline
Q1: Does your data set links to other data sets?                   \\ \hline
Q2: Do you provide provenance metadata?                            \\ \hline
Q3: Do you provide licensing metadata?                             \\ \hline
Q4: Do you use terms from widely deployed vocabularies?            \\ \hline
Q5: Are the URIs of proprietary vocabulary terms dereferenceable?  \\ \hline
Q6: Do you map proprietary vocabulary terms to other vocabularies? \\ \hline
Q7: Do you provide data set-level metadata?                        \\ \hline
Q8: Do you refer to additional access methods?                     \\ \hline
\end{tabular}
\caption{Criteria Group 4: Linked Data Publishing Checklist}
\label{tbl_cr_4_scala}
\end{table}

Since the whole paper is about Linked Data, it is important to analyse not only the solutions itself but also the resulting LDs. In order to do that, the Linked Data Publishing Checklist by Heath et.al.~\cite{heath2011linked} will be used. Another alternative could be the the LOD defintion itself by Tim Berners-Lee~\cite{berners2006linked}, but this paper will presume, that a L(O)D tool will produce valid L(O)D. As one can expect from a checklist, the rating for this criteria will be only fulfilled/not fulfilled.

The checklist can be seen in table~\ref{tbl_cr_4_scala}

\section{End Result}

Since this paper is designed to be used as a help for decisions, the aim can not be to find the "best" solution, therefore there will not be such a thing like an end result. E.g. for some situation a solution with bad usability can be more appropriate because of the supported data formats than a solution with a higher overall rating.

\begin{table}[htb]
\begin{tabular}{|l|l|}
\rowcolor[HTML]{EFEFEF} 
\hline
\textbf{Criteria group}             & \multicolumn{1}{c|}{\textbf{Criteria}} \\ \hline
\multirow{3}{*}{Group 1}            & Maintainability                        \\ \cline{2-2} 
                                    & Data Freshness                         \\ \cline{2-2} 
                                    & Flexibility                            \\ \hline
\multirow{5}{*}{Group 2: Usability} & Effectiveness                          \\ \cline{2-2} 
                                    & Efficiency                             \\ \cline{2-2} 
                                    & Satisfaction                           \\ \cline{2-2} 
                                    & Security                               \\ \cline{2-2} 
                                    & Learnability                           \\ \hline
Group 3: Data formats               & Data formats                           \\ \hline
\begin{tabular}[c]{@{}l@{}}Group 4: LD Publishing \\ Checklist\end{tabular}    & LD Publishing Checklist                \\ \hline
\end{tabular}
\centering
\caption{Complete Criteria Catalogue}
\label{tbl_cr_all}
\end{table}
\chapter{Comparison (RQ2 \& RQ3)}

It this section the comparison itself will be done. In order to do that, first in section~\ref{comp_classification} the classification introduced in section~\ref{classification} will be applied the found solution. Then in section~\ref{comparison} the criteria defined in section~\ref{criteria} will be applied for each of them, divided in the defined groups. The summary in section~\ref{comp_summary} will then give in overview of the done comparison. 

The found solutions can be seen as recap in table~\ref{tb:cmp_overview}, in order to simplify the following tables, each of the solutions is given an ID to refer.

\begin{table}[htbp]
\centering
\begin{tabular}{|l|l|}
\hline
\textbf{ID} & \textbf{Framework}    \\ \hline
1           & Euclid Project        \\ \hline
2           & LUCERO                \\ \hline
3           & Linked Data book      \\ \hline
4           & D2RQ Platform         \\ \hline
5           & Information Workbench \\ \hline
6           & \begin{tabular}[c]{@{}l@{}}Linked Data \\ Integration Framework\end{tabular}                  \\ \hline
7           & Eclipse RDF4J         \\ \hline
8           & Apache Jena           \\ \hline
\end{tabular}
\caption{Overview of the solutions}
\label{tb:cmp_overview}
\end{table}

\section{Classification}\label{comp_classification}

\begin{sidewaystable}[p]
\bigskip
\centering\small\setlength\tabcolsep{2pt}
\hspace*{-1cm}
\begin{tabular}{|c|c|c|c|c|c|l|}
\hline
\multicolumn{1}{|c|}{\textbf{\begin{tabular}[c]{@{}c@{}}Sol- \\ ution\end{tabular}}} & \multicolumn{1}{c|}{\textbf{\begin{tabular}[c]{@{}c@{}}Arch-\\ itecture\end{tabular}}} & \multicolumn{1}{l|}{\textbf{\begin{tabular}[c]{@{}l@{}}Full-\\ Stack\end{tabular}}} & \multicolumn{1}{l|}{\textbf{\begin{tabular}[c]{@{}l@{}}Present-\\ ation \\ layer\end{tabular}}} & \multicolumn{1}{l|}{\textbf{\begin{tabular}[c]{@{}l@{}}Business \\ Layer\end{tabular}}} & \multicolumn{1}{l|}{\textbf{\begin{tabular}[c]{@{}l@{}}Data \\ Access \\ Layer\end{tabular}}} & \textbf{Note} \\ \hline
1 & x &  &  & x & x &  \\ \hline
2 & x &  &  & x & x &  \\ \hline
3 & x &  &  & x & x &  \\ \hline
4 &  &  & x & x & x & \begin{tabular}[c]{@{}l@{}}D2R Server includes HTML \\ view and SPARQL endpoints\end{tabular} \\ \hline
5 &  & x &  &  &  &  \\ \hline
6 &  &  &  & x & x &  \\ \hline
7 &  &  &  & x & x &  \\ \hline
8 &  &  & x & x & x & \begin{tabular}[c]{@{}l@{}}Provides optionally SPARQL \\ endpoints and stand-alone \\ server with Jetty\end{tabular} \\ \hline
\end{tabular}
\hspace*{-1cm}
\caption{Classification}
\label{tbl:cmp_classification}
\end{sidewaystable}

\section{Comparison}\label{comparison}
\subsection{Criteria Group 1: Criteria from previous study}

\begin{table}[htbp]
\centering
\begin{tabular}{|c|l|l|l|l|}
\hline
\textbf{Solution}   & \textbf{Maintainability} & \textbf{Data Freshness} & \textbf{Flexibility} & \textbf{Note} \\ \hline
\textbf{1} &                          &                         &                      &               \\ \hline
\textbf{2} &                          &                         &                      &               \\ \hline
\textbf{3} &                          &                         &                      &               \\ \hline
\textbf{5} &                          &                         &                      &               \\ \hline
\textbf{5} &                          &                         &                      &               \\ \hline
\textbf{6} &                          &                         &                      &               \\ \hline
\textbf{7} &                          &                         &                      &               \\ \hline
\textbf{8} &                          &                         &                      &               \\ \hline
\end{tabular}
\caption{Comparison Criteria Group 1}
\label{tbl_comp_gr_1}
\end{table}

\subsection{Criteria Group 2: Usability}

\begin{table}[htbp]
\centering
\begin{tabular}{|c|l|l|l|l|l|l|}
\hline
\textbf{Solution} & \textbf{Effectiveness} & \textbf{Efficiency} & \textbf{Satisfaction} & \textbf{Security} & \textbf{Learnability} & \textbf{Note} \\ \hline
\textbf{1}        &                        &                     &                       &                   &                       &               \\ \hline
\textbf{2}        &                        &                     &                       &                   &                       &               \\ \hline
\textbf{3}        &                        &                     &                       &                   &                       &               \\ \hline
\textbf{5}        &                        &                     &                       &                   &                       &               \\ \hline
\textbf{5}        &                        &                     &                       &                   &                       &               \\ \hline
\textbf{6}        &                        &                     &                       &                   &                       &               \\ \hline
\textbf{7}        &                        &                     &                       &                   &                       &               \\ \hline
\textbf{8}        &                        &                     &                       &                   &                       &               \\ \hline
\end{tabular}
\caption{Comparison Criteria Group 2: Usability}
\label{tbl_comp_gr_2}
\end{table}

\subsection{Criteria Group 3: Data formats}

\begin{table}[htbp]
\centering
\begin{tabular}{|c|l|l|}
\hline
\rowcolor[HTML]{EFEFEF} 
\textbf{Solution} & \textbf{Data formats} & \textbf{Note} \\ \hline
\textbf{1}        &                       &               \\ \hline
\textbf{2}        &                       &               \\ \hline
\textbf{3}        &                       &               \\ \hline
\textbf{5}        &                       &               \\ \hline
\textbf{5}        &                       &               \\ \hline
\textbf{6}        &                       &               \\ \hline
\textbf{7}        &                       &               \\ \hline
\textbf{8}        &                       &               \\ \hline
\end{tabular}
\caption{Comparison Criteria Group 3: Data formats}
\label{tbl_comp_gr_3}
\end{table}

\subsection{Criteria Group 4: Linked Data Publishing Checklist}

\begin{table}[htbp]
\centering
\resizebox{\textwidth}{!}{%
\begin{tabular}{|l|l|l|l|l|l|l|l|l|}
\hline
\rowcolor[HTML]{EFEFEF} 
\textbf{Solution} & \textbf{1} & \textbf{2} & \textbf{3} & \textbf{4} & \textbf{5} & \textbf{6} & \textbf{7} & \textbf{8} \\ \hline
Does your data set links to other data sets? & x & x & x & x & x & x & x & x \\ \hline
Do you provide provenance metadata? & x & x & x & x & x & x & x & x \\ \hline
Do you provide licensing metadata? & x & x & x & x & x & x & x & x \\ \hline
Do you use terms from widely deployed vocabularies? & x & x & x & x & x & x & x & x \\ \hline
Are the URIs of proprietary vocabulary terms dereferenceable? & x & x & x & x & x & x & x & x \\ \hline
Do you map proprietary vocabulary terms to other vocabularies? & x & x & x & x & x & x & x & x \\ \hline
Do you provide data set-level metadata? & x & x & x & x & x & x & x & x \\ \hline
Do you refer to additional access methods? & x & x & x & x & x & x & x & x \\ \hline
\end{tabular}%
}
\caption{My caption}
\label{my-label}
\end{table}

\section{Summary}\label{comp_summary}
\chapter{Usage at TU Wien}
\chapter{Conclusion And Future Work}\label{ch:summary}

The overall goal of this thesis was to compare common LOD solutions and give TU Wien a guideline for choosing and developing such an application. The concrete research question was: 

\textbf{RQ:} \textit{How do common LOD solutions compare against each?}
\begin{enumerate}
\item \textbf{RQ1:} What are existing LOD solutions?
\item \textbf{RQ2:} What are criteria to compare solutions?
\item \textbf{RQ3:} How do they compare against each other?
\item \textbf{RQ4:} What can be a solution for TU Wien?
\end{enumerate}

The first part of this thesis investigated the first research question by conducting a literature study and discussing the term "framework" and why it was not used instead of the term "solution" (see subsection~\ref{def:framework}). The methodology can be seen in chapter~\ref{chap:methodology}, the results in section~\ref{overview}.

In the second part (section~\ref{criteria}, a set of criteria was developed in order to compare the found solutions and answer research question 2. These criteria were then used to investigate the found solutions under the aspects of them. The results can be found in section~\ref{comparison}.

In the final part, the found solutions were examined on the usefulness for TU Wien, in section~\ref{ch:tuwien}.  

\section{Conclusion}

In the following subsections each research question will be addressed and revisited :

\subsection{Existing LOD solutions}

Using the method of a literature review, eight solution were found for this work: 

\begin{itemize}
\itemsep0pt
\item Euclid Project
\item LUCERO
\item Linked Data Book
\item D2RQ Platform
\item LDIF
\item Eclipse RDF4J
\item Apache Jena
\end{itemize}

Of the found candidates a few were discarded since they were not (public) documented or too generic.

\subsection{Comparision Criteria}

Since the found solutions showed a high variation, it was necessary to find a set of criteria which can be partially not applicable while still meaningful in their entirety. Therefore four criteria group were defined: \textit{Criteria from a Previous Study, Usability, Data formats and the Linked Data Checklist}. Since the aim of this thesis is not to find the "best" solution, the criteria were not weighted and had only a scala as assessment instead of e.g. a point system.

\subsection{Comparision}

As expected, some criteria (especially Usability) were not applicable for some solutions, but the concept of multiple criteria groups worked out. The results (an overview of the full results can be seen in the Appendix) in short are:


\begin{itemize}

\item The Euclid Project was too generic to give a definite evaluation for the most criteria, since most of it is depending on an actual implementation. Correctly applied it does however supports directly or indirectly maintainability, data freshness, flexibility, various kinds of data formats and the complete LD checklist.

\item LUCERO was found outdated and bad documented, it is not recommended to use it (any more).

\item Similar to Euclid, the Linked Data Book was too generic to find explicit assessments for the criteria, too much is depending on the actual implementation.

\item The D2RQ platform has (due its nature as specialised tool) only limited support for data formats, but received a good evaluation in the other criteria groups.

\item As an all-in-one solution the Information Workbench has good results in all groups. It has to be noted, that this tool requires licensing when used outside an educational scope.

\item LDIF is due its concept on the one side a very flexible framework for handling different data sources with a wide range of vocabulary. On the other hand the specialisation leads to a specific focus resulting in a limit number of supported data formats. It is recommended to use this solution in specialised situation or in combination with other solutions like Jena or RDF4J.

\item Eclipse RDF4J and Apache Jena both are very different to the other solutions, they are Java frameworks, requiring to write code against their API. For both the results of the comparison highly depend on the actual implementation.
\end{itemize}

\subsection{Solution For TU Wien}

For the TU Wien a set of situations were developed in order to find a context for which a solution can be recommended. It was found, that the "ideal" solution, based on the given situations, requirements and stakeholders, would be an self-developed platform, using either Jena or RDF4J (or a similar tool) while using the Euclid architecture as a blueprint and LDIF if a mapping is necessary.

\section{Future Work}

It is recommended to extend the list of possible solutions and applying the given set of criteria in order to find a suitable solution for TU Wien. The found combination of Jena/RDF4J with Euclid and LDIF might be a way of developing a platform, but it is not necessary to use this \emph{exact} combination. Nevertheless it should be keep in mind, that this way is more extensive than using e.g. the Information Workbench. But as an advantage, the full platform can be controlled.
\chapter{Appendix}

\begin{table}[htbp]
\centering
\resizebox{\textwidth}{!}{%
\begin{tabular}{|l|c|c|c|c|c|l|}
\hline
\rowcolor[HTML]{EFEFEF} 
\textbf{ID} & \multicolumn{1}{c|}{\textbf{\begin{tabular}[c]{@{}c@{}}Arch-\\ itecture\end{tabular}}} & \multicolumn{1}{l|}{\textbf{\begin{tabular}[c]{@{}l@{}}Full-\\ Stack\end{tabular}}} & \multicolumn{1}{l|}{\textbf{\begin{tabular}[c]{@{}l@{}}Present-\\ ation \\ layer\end{tabular}}} & \multicolumn{1}{l|}{\textbf{\begin{tabular}[c]{@{}l@{}}Business \\ Layer\end{tabular}}} & \multicolumn{1}{l|}{\textbf{\begin{tabular}[c]{@{}l@{}}Data \\ Access \\ Layer\end{tabular}}} & \textbf{Note} \\ \hline
\textbf{\begin{tabular}[c]{@{}c@{}}Euclid \\Project\end{tabular}} & x & & & x & x & \\ \hline
\textbf{LUCERO} & x & & & x & x & \\ \hline
\textbf{\begin{tabular}[c]{@{}c@{}}Linked\\ Data\\ book\end{tabular}} & x & & & x & x & \\ \hline
\textbf{\begin{tabular}[c]{@{}c@{}}D2RQ \\Platform\end{tabular}} & & & x & x & x & \begin{tabular}[c]{@{}l@{}}D2R Server includes HTML \\ view and SPARQL endpoints\end{tabular} \\ \hline
\textbf{\begin{tabular}[c]{@{}c@{}}Information \\Workbench\end{tabular}} & & x & & & & \\ \hline
\textbf{LDIF} & & & & x & x & \\ \hline
\textbf{\begin{tabular}[c]{@{}c@{}}Eclipse \\RDF4J\end{tabular}} & & & & x & x & \\ \hline
\textbf{\begin{tabular}[c]{@{}c@{}}Apache \\Jena\end{tabular}} & & & x & x & x & \begin{tabular}[c]{@{}l@{}}Provides optionally SPARQL \\ endpoints and stand-alone \\ server with Jetty\end{tabular} \\ \hline
\end{tabular}
}
\caption{Classification}
\label{app:cmp_classification}
\end{table}

\begin{sidewaystable}
\centering
\begin{tabular}{|l|c|c|c|c|c|c|c|c|}
\hline
\rowcolor[HTML]{EFEFEF} 
\textbf{ID} & \textbf{\begin{tabular}[c]{@{}c@{}}Maintain-\\ ability\end{tabular}} & \textbf{\begin{tabular}[c]{@{}c@{}}Data \\ Freshness\end{tabular}} & \textbf{Flexibility} & \textbf{\begin{tabular}[c]{@{}c@{}}Effective-\\ ness\end{tabular}} & \textbf{Efficiency} & \textbf{\begin{tabular}[c]{@{}c@{}}Satis- \\ faction\end{tabular}} & \textbf{Security} & \textbf{\begin{tabular}[c]{@{}c@{}}Learn-\\ ability\end{tabular}} \\ \hline
\textbf{\begin{tabular}[c]{@{}c@{}}Euclid \\Project\end{tabular}} & + & yes & yes & N.A. & N.A. & N.A. & - & + \\ \hline
\textbf{LUCERO} & ? & yes & yes & N.A. & N.A. & N.A. & - & - \\ \hline
\textbf{\begin{tabular}[c]{@{}c@{}}Linked\\ Data\\ book\end{tabular}} & + & yes & yes & N.A. & N.A. & N.A. & - & + \\ \hline
\textbf{\begin{tabular}[c]{@{}c@{}}D2RQ \\Platform\end{tabular}} & + & yes & yes & + & + & + & - & + \\ \hline
\textbf{\begin{tabular}[c]{@{}c@{}}Information \\Workbench\end{tabular}} & ? & yes & yes & + & + & + & + & + \\ \hline
\textbf{LDIF} & + & yes & yes & N.A. & N.A. & N.A. & - & + \\ \hline
\textbf{\begin{tabular}[c]{@{}c@{}}Eclipse \\RDF4J\end{tabular}} & $\sim$ & ? & yes & N.A. & N.A. & N.A. & $\sim$ & + \\ \hline
\textbf{\begin{tabular}[c]{@{}c@{}}Apache \\Jena\end{tabular}} & $\sim$ & ? & yes & N.A. & N.A. & N.A. & + & + \\ \hline
\end{tabular}
\caption{Comparison group 1 \& 2}
\label{app:gr_1_2}
\end{sidewaystable}

\begin{sidewaystable}
\centering
\begin{tabular}{|l|l|c|c|c|c|c|c|c|c|}
\hline
\rowcolor[HTML]{EFEFEF} 
\textbf{ID} & \multicolumn{1}{c|}{\cellcolor[HTML]{EFEFEF}\textbf{\begin{tabular}[c]{@{}c@{}}Data \\ formats\end{tabular}}} & \textbf{Q1} & \textbf{Q2} & \textbf{Q3} & \textbf{Q4} & \textbf{Q5} & \textbf{Q6} & \textbf{Q7} & \textbf{Q8} \\ \hline
\textbf{\begin{tabular}[c]{@{}c@{}}Euclid \\Project\end{tabular}} & \begin{tabular}[c]{@{}l@{}}Potential every \\ possible format\end{tabular} & x & x & x & x & x & x & x & x \\ \hline
\textbf{LUCERO} & \begin{tabular}[c]{@{}l@{}}Default RSS \& \\ XML\end{tabular} & x & ? & x & x & x & x & x & x \\ \hline
\textbf{\begin{tabular}[c]{@{}c@{}}Linked \\Data \\book\end{tabular}} & \begin{tabular}[c]{@{}l@{}}Recipes for RDF, \\ XML, HTML, \\ relational \\ databases, \\ Wrapper\end{tabular} & x & ? & x & x & x & x & x & x \\ \hline
\textbf{\begin{tabular}[c]{@{}c@{}}D2RQ \\Platform\end{tabular}} & \begin{tabular}[c]{@{}l@{}}Only relational \\databases\end{tabular} & x & x & x & x & x & x & ? & ? \\ \hline
\textbf{\begin{tabular}[c]{@{}c@{}}Information \\Workbench\end{tabular}} & \begin{tabular}[c]{@{}l@{}}Table-based (csv, \\ excel, groovy,jdbc, \\ rest, tsv, sparql \\ etc), tree-based \\ (xml, json, etc), \\ RDF\end{tabular} & x & x & x & x & x & x & ? & ? \\ \hline
\textbf{LDIF} & \begin{tabular}[c]{@{}l@{}}N-Quads dumps, \\ RDF/XML,N-\\ Triples, Turtle \\ dumps, \\ dereferenced\\ URIs, SPARQL\end{tabular} & x & x & x & x & x & x & - & - \\ \hline
\textbf{\begin{tabular}[c]{@{}c@{}}Eclipse \\RDF4J\end{tabular}} & Only RDF & x & ? & x & x & x & x & x & x \\ \hline
\textbf{\begin{tabular}[c]{@{}c@{}}Apache \\Jena\end{tabular}} & \begin{tabular}[c]{@{}l@{}}Only RDF \& \\ OWL\end{tabular} & x & - & x & x & x & x & ? & ? \\ \hline
\end{tabular}
\caption{Comparison group 3 \& 4}
\label{app:gr_3_4}
\end{sidewaystable}

\begin{table}[hbtp]
\centering
\resizebox{\textwidth}{!}{%
	\begin{tabular}{|l|l|l|}
		\hline
		\rowcolor[HTML]{EFEFEF} 
		\textbf{Scenario} & \textbf{Requirement} & \textbf{Solution} \\ \hline
		%%
		\textbf{\begin{tabular}[c]{@{}l@{}}Specialised \\ Single Solution\end{tabular}} & 
		\begin{tabular}[c]{@{}l@{}}
			-) small scale \\ 
			-) small number of data sets \\ 
			-) specialised solution \\ 
			-) simple
		\end{tabular} & 
		\begin{tabular}[c]{@{}l@{}}
			-) Option 1: D2RQ \\ 
			-) Option 2: Jena + Euclid \\ (+ LDIF)
		\end{tabular} \\ \hline
		%%
		\textbf{\begin{tabular}[c]{@{}l@{}}Function-rich \\ Platform\end{tabular}} & 
		\begin{tabular}[c]{@{}l@{}}
			-) various kinds of data sets \\ (in number and formats) \\ 
			-) medium scale \\ 
			-) platform with additional features \\ 
			-) easy to use/implement
		\end{tabular} & 
		Information Workbench \\ \hline
		%%
		\textbf{\begin{tabular}[c]{@{}l@{}}Complete \\ Controlled \\ Platform\end{tabular}} & 
		\begin{tabular}[c]{@{}l@{}}
			-) various kinds of data sets \\ (in number and formats) \\ 
			-) medium scale \\ 
			-) platform with additional features \\ 
			-) no licensing \\
			-) Open Source
		\end{tabular} & 
		\begin{tabular}[c]{@{}l@{}}
			Jena/RDF4J + Euclid \\ (+ LDIF)
		\end{tabular} \\ \hline
	\end{tabular}%
}
\caption{Summary of Scenarios \& proposed solutions for TU Wien}
\end{table}

%%%%%%%%%%%%%%%%%%%%%%%%%%%%%%%%%%%%%%%%%%%%%%%%%%%%%%%%%%%%%%

\backmatter

%%%%%%%%%%%%%%%%%%%%%%%%%%%%%%%%%%%%%%%%%%%%%%%%%%%%%%%%%%%%%%

% Use an optional list of figures.
\listoffigures % Starred version, i.e., \listoffigures*, removes the toc entry.

% Use an optional list of tables.
%\cleardoublepage % Start list of tables on the next empty right hand page.
\listoftables % Starred version, i.e., \listoftables*, removes the toc entry.

%%%%%%%%%%%%%%%%%%%%%%%%%%%%%%%%%%%%%%%%%%%%%%%%%%%%%%%%%%%%%%

% Add a bibliography.
%\bibliographystyle{alpha}
%\bibliography{bibliography}

%\addcontentsline{toc}{chapter}{Appendix}

\printbibliography[title={References to refereed scientific work}, keyword=refsw]

\printbibliography[title={References to non-refereed work}, keyword=nonrefsw]

\printbibliography[title={References to websites}, type=misc]

\end{document}